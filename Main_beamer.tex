\documentclass{beamer}
\usetheme{Madrid}
\usepackage[framechange]{colourchange}
\usepackage[utf8]{vietnam}
\usepackage{vnlipsum}
\usepackage{contour} % Gói lệnh làm chữ nổi
\usepackage{multicol} % Sử dụng cho mục đích mục lục dài

% Dùng cho mục đích nội dung slide quá dài và qua trang
\setbeamertemplate{frametitle continuation}[from second][(tiếptục)]
\setbeamertemplate{sectionintoc}[ballnumbered]
\setbeamertemplate{subsectionintoc}[subsectionsnumbered]

% ĐỊNH NGHĨA VỀ SỐ TRÊN TIÊU ĐỀ 
\newcounter{chuong}
\setcounter{chuong}{1}
\newcommand{\oneline}{\par}
\newcommand{\setlinetitle}[1]{
	\renewcommand{\oneline}{#1}}
\newcommand{\tds}{\ifnum\thesection>0
	\textbf{\thesection.~\insertsection}
	\ifnum\thesubsection>0\oneline\normalsize
	\textbf{\thesection.\thesubsection.~\insertsubsection}\fi
	\fi}
\newcommand{\tdc}{\ifnum\thesection>0\textbf{\thechuong.
		\thesection.~\insertsection}
	\ifnum\thesubsection>0\oneline\normalsize\textbf{
		\thechuong.\thesection.\thesubsection.~\insertsubsection}
	\fi\fi}

% MÔI TRƯỜNG ĐỊNH NGHĨA, ĐỊNH LÝ, ...
\theoremstyle{plain} % làm chữ nghiêng.
\newtheorem{dl}{\textbf{Định lý}}
\numberwithin{dl}{section}
\newtheorem{md}{\textbf{Mệnh đề}}
\newtheorem{bd}{\textbf{Bổ đề}}

\theoremstyle{definition} % làm chữ đứng bình thường.
\newtheorem{dn}{\textbf{Định nghĩa}}
\numberwithin{dn}{section}
\newtheorem{vd}{Ví dụ}
\newtheorem{hq}{\bf Hệ quả}
%%%%%%%%%%
\setbeamertemplate{theorems}[numbered]


% Đặt tiêu đề cho bài báo cáo
\title[Tiêu đề của bài báo cáo]{
	{\color{yellow} \contourlength{2pt} \contour[10]{red}{Nguyễn Quốc Dương}}\\
	\textbf{MẪU BEAMER CHO BÁO CÁO KHOA HỌC\\
		BẰNG LATEX}
}

% Thông tin về tác giả của bài báo cáo, hoặc tên người hướng dẫn
\author{Nguyễn Quốc Dương}
\institute[Đại học Quy Nhơn]{\textbf{Khoa học dữ liệu ứng dụng}\\
	\textbf{Đại học Quy Nhơn}}
\date{Quy Nhơn, \today}


\begin{document}
% SLIDE TRANG BÌA
\frame{
\begin{center}
\textbf{BỘ GIÁO DỤC VÀ ĐÀO TẠO}\\
\textbf{{\large TRƯỜNG ĐẠI HỌC QUY NHƠN}}
\end{center}
\maketitle
}

% SLIDE MỤC LỤC 
\begin{frame}{Nội dung}
\AtBeginSubsection[]{
	\begin{frame}{Nộidung}
		\tableofcontents[currentsection,currentsubsection]
	\end{frame}
}
\end{frame}

\section{Giới thiệu}


\begin{frame}{\thesection . Giới thiệu}
	\begin{dn}
		contesdsdnt...
	\end{dn}
	\begin{dl}
		vv
	\end{dl}
\begin{itemize}
\item Seamus Bradley viết gói lệnh colourchange.styđể thay đổi mầu
mỗi trang beamer.
\item https ://ctan.org/pkg/ colourchange

\item Sau đây là các kỹ thuật dùng gói này:
\end{itemize}
\end{frame}

	
\selectmanualcolour{red}
\begin{frame}{Trang mầu đỏ}
	
{\color{yellow} \contourlength{2pt} \contour[20]{blue}{Nguyễn Quốc Dương}}

\end{frame}

\selectmanualcolor{green}
\begin{frame}{Trang mầu nõn chuối}
	sd
\end{frame}
\selectmanualcolor{blue}
\begin{frame}{Trang mầu xanh}
	sds
\end{frame}

\end{document}
